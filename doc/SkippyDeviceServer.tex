\documentclass[a4paper,10pt]{article}
\usepackage[utf8x]{inputenc}

\usepackage[left=2.5cm,right=2.5cm,top=2cm,bottom=2cm]{geometry}
\usepackage{amsmath,amsfonts,amssymb}
\usepackage{graphicx}
\usepackage[colorlinks=true,linkcolor=blue,urlcolor=blue,citecolor=blue]{hyperref}
\usepackage{color}
\usepackage{listings}

%opening
\title{Skippy Device Server}
\author{Sergi Blanch-Torn\'e\\\small{Controls Software Engineer - Alba Synchrotron}\\{\tt \small{sblanch@cells.es}}}

\newcommand{\todo}[1]{\texttt{\color{red}TODO:} ``\emph{#1}''}
\newcommand{\fixme}[1]{\texttt{\color{red}FIXME:} ``\emph{#1}''}
\newcommand{\ok}[1]{``#1'' [\texttt{\color{green}OK}]}

\begin{document}

\maketitle

\begin{abstract}
Following the \href{https://en.wikipedia.org/wiki/Standard_Commands_for_Programmable_Instruments}{wikipedia}'s definition:  the SCPI ``\emph{defines a standard for syntax and commands to use in controlling programmable test and measurement devices}''. Many scientific instrumentation uses this schema for the configuration and operation of the instrument of its purpose.

Alba has used an specific Device Server, under the name \emph{PyVisaInstrWrapper}, to communicate with instruments like scopes, radio frequency generators, arbitrary signal generators and spectrum analyser. This device was an extension from the original PyScope, becoming it one of the internal device classes. Many improvements has been introduced to this Device Server, but nowadays it is showing the limit of this design. A complete refactoring is needed.
\end{abstract}

\section{Design}

The main restriction that the PyVisaInstrWrapper shows is the dependency on the PyVisa subdevice to manage the communications with the instrument. Also the growing number of query commands to the instrument has reach the limit of the basic design.

\subsection{Machine state}

\begin{figure}[h]
    \centering{
         \includegraphics[width=0.75\textwidth]{StateMachine.png}
         \caption{State machine diagram of the Device in the Skippy Device Server} \label{fig:stateMachine}
    }
\end{figure}

\todo{explain the figure \ref{fig:stateMachine}}

\subsection{Device Properties}

\begin{itemize}
    \item {\tt Instrument}: mandatory property to configure the instrument that will be introduced to the distributed system via the device.
    \begin{itemize}
        \item If the content string is a host name, use sockets to communicate with the instrument
        \item If the content string is a device name and the class of it is a PyVisa, use it as the bridge to the instrument.
        \item else, decay to fault state with the appropriate status message
    \end{itemize}
    \item {\tt Port}: optional property to specify a port, in the socket communication, when it should be different than the default 5025.
    \item {\tt NumChannels}: optional property to define, in case the instrument has channels, the number of them available.
    \item {\tt NumFunctions}: optional property to define, in case the instrument has functions, the number of them available.
    \item {\tt MonitoredAttributes}: When the device is in RUNNING state, the attributes listed here will be monitored (having events) with a period said in the attribute TimeStampsThreashold (or different if specified with a : separator after the attrName).
    \item {\tt AutoStandby}: When device startup, try an standby() to connect to the instrument authomatically. Default True.
    \item {\tt AutoOn}: When device startup, try an on() begin communication with the instrument. Default True.
    \item {\tt AutoStart}: When device startup, try an Start() to launch the necessary monitor threads (if {\tt MonitoredAttributes} is configured), authomatically. Default True.
\end{itemize}

\subsection{Device Commands}

\begin{itemize}
    \item {\tt IDN()}: Request identification to the instrument.
    \item {\tt Off()}: Release the communication with the instrument.
    \item {\tt Standby()}: Open the communications with the instrument and do the identification and attribute builder, but do not allow yet any other query.
    \item {\tt On()}: Stablish communication with the instrument.
    \item {\tt Start()}: Start an active monitoring.
    \item {\tt Stop()}: Stop the active monitoring.
    \item {\tt AddMonitoring(AttrName)}: Add an attribute to the list of monitored attributes
    \item {\tt RemoveMonitoring(AttrName)}: Remove an attribute from the list of monitored attributes
    \item {\tt SetMonitoringPeriod([AttrName,AttrPeriod])}: From the list of already monitored attributes, stablish (or change) the period that it is checked. With this command the monitored attribute will become one of the splitted thread monitor, even if the period in the same thant the normal periodic. To force to place in this normal monitor list, use period value $0$.
    \item {\tt GetMonitoringPeriod(AttrName)}: Get the period that is checked an attribute monitored. {\tt NaN} if it is not monitored.
    \item {\tt \emph{Exec():}} Expert attribute to look inside the device during execution.
    \item {\tt \emph{CMD():}} Expert command for a direct send of a SCPI command and read the answer.
    \item {\tt \emph{CMDfloat():}} Expert command for a direct send of a SCPI command and read the answer converted to a float list.
    \item \todo{{\tt \emph{DumpAttr([file,time]):}} Once received, dump the raw data readed from the read of an attribute during the specified seconds}
\end{itemize}

\subsection{Device Attributes}

\begin{itemize}
    \item {\tt idn}: read attribute with the identification of the instrument linked.
    \item {\tt QueryWindow}: Expert attribute to configure the number of request sent in parallel to the instrument. Bigger queries will be splitted in subqueries o of this size.
    \item {\tt TimeStampsThreshold}: This value sets the threshold time to use a cached value or hardware read it.
\end{itemize}

Other attributes can be built using a \href{http://en.wikipedia.org/wiki/Builder_pattern}{Builder pattern}. In fact, when the device starts and the instrument is identified, with the information provided in this identification, the device can find how to build the instructions set.

An example of an attribute definition:

\begin{lstlisting}[language=python,basicstyle=\footnotesize]
Attribute('State',
          {'type':PyTango.CmdArgType.DevBoolean,
           'dim':[0],
           'readCmd':lambda ch,num:":%s%d:DISPlay?"%(ch,num),
           'writeCmd':lambda ch,num:(lambda value:":%s%d:DISPlay %s"%(ch,num,value)),
           'channels':True,
           'functions':True,
         })
\end{lstlisting}

This example, in a oscilloscope with 4 channels and 4 functions, will setup 8 {\tt READ\_WRITE} boolean attributes, called with the pattern: {\tt State\{Ch,Fn\}[1..4]}. Point to comment is the {\tt readCmd} and the {\tt writeCmd} and the use of {\tt lambda}s. In the \emph{read} case, and because this is an attribute description to build more than one, the {\tt lambda} function is set to build the final command for each of the channels and functions. A bit more complicated is the \emph{write} case, where there are two nested {\tt lambda}s, one used in the attribute build and the other used when write the attribute because is when the value is known.

Another example of an attribute definition:

\begin{lstlisting}[language=python,basicstyle=\footnotesize]
Attribute('Frequency',
          {'type':PyTango.CmdArgType.DevDouble,
           'dim':[0],
           'readCmd':":FREQ?",
           'writeCmd':lambda value:":FREQ %s"%(str(value)),
           'rampeable':True,
         })
\end{lstlisting}

That example, in a RF generator, will setup 3 {\tt READ\_WRITE} double attributes, called {\tt Frequency}, {\tt FrequencyStep}, {\tt FrequencyStepSpeed}. This describes a behabiour where, when a \emph{frequency} is set, a thread will be launched to each \emph{StepSpeed} seconds, the current value will be increased/decreased by the \emph{Step} in the direction of the setpoint.

Here an example of an spectrum attribute definition:

\begin{lstlisting}[language=python,basicstyle=\footnotesize]
Attribute('Waveform',
          {'type':PyTango.CmdArgType.DevDouble,
           'dim':[1,40000000],
           'readCmd':lambda ch,num:":WAVeform:SOURce %s%d;:WAVeform:DATA?"%(ch,num),
           'channels':True,
           'functions':True,
         })
\end{lstlisting}

With this attribute is broken a backward compatibility with PyVisaInstrWrapper because there those spectrum attributes did not follow the naming of name. There is a naming convention in attribute definitions with channels and functions, to the name given in the definition is concatenated at the end two characters ({\tt Ch} or {\tt Fn}) followed by the number of the channel.

\subsection{Class Diagram}

In figure \ref{fig:classDiagram} can be found a UML draw with the class diagram.

\begin{figure}[h!]
    \centering{
         \includegraphics[width=\textwidth]{ClassDiagram.png}
         \caption{Class diagram of the Skippy device.} \label{fig:classDiagram}
    }
\end{figure}

\subsection{Device features}

\begin{itemize}
    \item Device is capable to dynamically build attributes based on the instrument identification (described in the the particular instruction set).
    \begin{itemize}
        \item Supported Scalar and Spectrum (1 dimension arrays) attribute definition. Image (2 dimension arrays) are not currently developed, neither in the schedule because they are not needed by any of the current supported instruments. Although the device is prepared to sopport them if need be. (Perhaps the frequencies spectra in SpectrumAnalyser).
        \item The dynamic attributes build after the instrument identification can be Memorized if this has been configured this way in the file with the instruction set.
        \item Reconnect after network cut (or PyVisa)
    \end{itemize}
    \item Multiple request of data to the instrument: As figure \ref{fig:groupedRequests} describes, when the device receives a request for reading to the instrument it can manage to cut those requests in subsets to avoid stress in the instrument. The communication itself supports a direct socket connection to the instrument and the use of the PyVisa device as a bridge in the communication.
\end{itemize}

\begin{figure}[h]
    \centering{
         \includegraphics[width=0.75\textwidth]{groupedRequests.png}
         \caption{multiple request of readings to the instrument.} \label{fig:groupedRequests}
    }
\end{figure}

\begin{itemize}
    \item Write process to some attributes would require a smooth ramp. As an example, to change the frequency in the RFGenerator, the change would not be set directly (or the timing system will suffer). Then in the build process of an specific attribute with a certain flag of "movability" (name to be determined) two extra attributes, internals in the device, must be built also: a Step and a StepSpeed. On the write operation a thread should be launched and every "StepSpeed" seconds change the value by the content in "Step" in the direction to the final setpoint. This was already implemented in PyRfSignalGenerator Class.
    \item Monitor attributes. Having a property with a list of attributes, if they exist in the instrument definition, configure them in events and do a separated thread polling that will emit events on this attributes. This feature is different that setting up a "polling" from tango to the attribute because this fails to use the multiple reading.
    \begin{itemize}
        \item In the definition of the attribute to be monitored, the \emph{MonitoredAttributes} property, using an specific notation (a ':' followed by a number of seconds) can set up an specific monitoring period for it in particular.
        \begin{itemize}
            \item When the monitoring is by the specific period, different attributes with the same period, should be read by the same monitor thread.
        \end{itemize}
        \item A command way feature is required to add and remove elements from this monitoring list. Together with a set of commands to setup specific monitoring periods.
        \item \fixme{When an attribute is being monitored, a {\tt read\_attr()} should return cached value and not force a hardware read.}
    \end{itemize}
    \item \todo{Once this monitoring feature is available, other attributes not monitored that have reached some reading frequency, would be included in the monitoring loop (not emitting events) in order to reduce the load of the readings. The frequency reading should be monitored to notice when this has reduced over another threshold to avoid unnecessary readings.}
    \item \fixme{When many attributes are being read, and not by {\tt read\_attributes([])}, prioritise. (This happens, for example, when opens the device with \emph{Atkpanel}) }
\end{itemize}

\subsection{Known bugs}

\begin{itemize}
    \item \fixme{It have seen, when the read frequency is very fast, like read a waveform above $5Hz$, some exceptions that suggest a responses cross between spectrum and scalar. This is potentially critical, because is undetectable if it happens between scalars.}
    \item \fixme{When a function is not defined (not when if off like a channel, this is like ask for a channel that doesn't exist physically), the device hangs due to an exception from the instrument and a very long time out.}
\end{itemize}

\end{document}
